\section{REVIEW OF EXISTING ALTERNATIVES}
                This chapter should be able to present the following in not more than 600 words: description of how users/clientele currently cope with the problem; assessment of the best available resources for addressing the problem; and description of how your project stands out in comparison to the existing alternatives.
			\subsection{How to cite references}
				\lipsum[4]
				\subsubsection{Apa Subsubsection}
					APA CITATION SAMPLE HERE USING BIBTEX \citep{CrescenziKann1997}
			\subsection{Why do we need this?}
    			The review of existing alternatives is a thorough study of existing applications or solutions and literature that are related to your project. We do this for the following reasons:
                    \begin{itemize}
                        \item you get to see what other professionals or researchers and authors in the area where your topic falls have done or are currently doing;
                        \item you will have a deeper understanding of your topic and of the application that you are proposing to develop;
                        \item you will know if someone else has already done the same (if you are duplicating a project or a research); and
                        \item you will have an idea on how other people solved the problems that are similar to yours.
                    \end{itemize}

			% ------------------------
    % Instructions
    % ------------------------
    Instructions: Fill in the table below with your proposed system and at least 3 related systems. 
    Be concise and critical. After completing the table, answer the reflection questions. Once all your answers are in place, remove the instructions and examples before compiling and submitting.}
    
    % ------------------------
    % Comparison Table
    % ------------------------
    \begin{table}[ht]
    \vspace{4ex}
    \centering
        \caption{Summary of Existing Alternatives}   
        \label{table:existing-alternatives-summary}
        \smalltable
        % Use ragged-right paragraph columns so cell text is left-aligned and wraps
        \begin{tabular}{|>{\raggedright\arraybackslash}p{2.8cm}|>{\raggedright\arraybackslash}p{1.8cm}|>{\raggedright\arraybackslash}p{2.8cm}|>{\raggedright\arraybackslash}p{2.8cm}|>{\raggedright\arraybackslash}p{2.8cm}|}
            \hline
            	\textbf{System (License)} & \textbf{Platform} & \textbf{Brief Description} & \textbf{Basic Functionalities} & \textbf{Relation to the Proposed System} \\
            \hline\hline
            	\textbf{Proposed System: StudyBuddy AR (to be determined)} & Mobile (Android/iOS) & AR app for visualizing academic concepts across disciplines & View molecules, anatomy, circuits, and diagrams in AR; interactive learning modules & -- \\
            \hline
            Google Expeditions (Free but discontinued) & Mobile (Android/iOS) & AR/VR app for virtual field trips and basic concepts & Immersive AR/VR lessons and tours & Both use AR for education but Expeditions is broad and discontinued \\
            \hline
            Anatomy 3D Atlas (Paid - Subscription) & Mobile (Android/iOS) & Interactive 3D atlas for studying human anatomy & Layered 3D models of organs and systems & Similar in 3D visualization but limited to medicine \\
            \hline
            AR Circuits (Free) & Mobile (Android) & Educational AR app for visualizing circuit components & Place circuit components in AR and test connections & Similar AR approach but restricted to electronics \\
            \hline
        \end{tabular}
    \vspace{4ex}
    \end{table}

    Discuss the details of each existing alternative and this question: Can You Use This Instead? Why or Why Not? In addition, discuss why your proposed system is necessary, focusing on the following.
    \begin{itemize}
    \item Biggest gap in existing systems: Example: No existing AR learning app provides a single platform that covers multiple subjects. Most are subject-specific.
    \item Why our project is still worth developing: Example: StudyBuddy AR combines features from various niche applications into one unified platform, making it more versatile for students.
    \item Our unique contribution: Example: A cross-disciplinary AR educational tool designed for general college learning, not restricted to one field.
\end{itemize}
