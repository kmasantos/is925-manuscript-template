\section{INTRODUCTION}
    \subsection{Background of the project}

            One of the most common names given to dogs in the Philippines is \textit{Bantay}, which means to guard. This is rooted in how they were originally envisioned to be—the defender of the house, the first line of defense against burglars, and the one to take the bullet.

            In the first quarter of 2023, Social Weather Stations (SWS), a prominent social research institution in the Philippines, conducted a survey to assess the prevalence of pet ownership among Filipino families. The survey revealed that a significant 64 percent of households own pets, with 59 percent of these households keeping dogs and/or cats. The remaining 5 percent of pet owners have other species of animals. In addition to these findings, the Philippine Animal Welfare Society (PAWS) reported that as of 2019, there were an estimated 12 million stray dogs in the country. These animals are either abandoned family pets or the offspring of dogs that reproduce on the streets.
            
            Fortunately, the way people view animals, especially those who coexist with us inside our homes, is slowly changing. Because of the same problem of animal neglect and cruelty, many private individuals established their non-profit organizations (NGOs) and have devoted their efforts to the holistic rescue, rehabilitation, and rehoming of these once neglected animals. However, not all NGOs are created equal.
            
            NGOs that have established their name over the years tend to receive more social media traffic and popularity, leading to increased funding from private companies and support from citizens. In contrast, some organizations, despite having one of the largest numbers of rescues, barely get attention. Day in and day out, they rely on Facebook Live and donation requests. Month after month, they face mounting debts.
            
            This study focuses on extending technological advancements to provide a more inclusive platform for the rescue community—not just for specific NGOs, but for any sanctuary in need.

            


    \subsection{Objectives}
    
        This project aims to provide NGOs with a web platform where they can create and maintain rescue profiles, and offer general users a matchmaking-style mobile platform enhanced with machine learning to connect rescued animals with potential adopters or sponsors. The system will prioritize profile recommendation based on user interactions with various rescue profiles, tailoring matches to those who engage most strongly with specific animal characteristics or story. The platform will empower rescue organizations of all sizes to share their stories, solicit support, and accept sponsorship, while also facilitating connections with prospective adoptive families. The matching process will evolve over time, refining the algorithm based on user interactions. This approach ensures that adoption opportunities are not limited to larger, more established NGOs, but are accessible to organizations of all sizes.
        
    \subsection{Scope and Limitations}
        This study primarily focuses on the design and implementation of a multi-platform product aimed at helping Philippine-based animal welfare groups better connect with the rescue community for sponsorship and adoption.
        \subsubsection{Scope}
             The project includes the development of a web application where NGOs can sign up and create animal profiles. Once animal profiles are published, they will automatically be made available on the mobile app, where users can view recommended animal profiles based on their preferences. When a user swipes right on an animal profile, it will be saved to their matches, and they can view additional details, such as the NGO the animal belongs to, as well as access the adoption application and sponsorship options.
             
             For sponsorship and other payments, the mobile platform will accept payments via Paymongo, limited to one-time credit card transactions.
        \subsubsection{Limitations}
            All other operations requiring manual intervention are out of scope, including but not limited to: establishing a bank account and Paymongo profile for donation disbursement, conducting interviews with adoption applicants, arranging in-person meetups, obtaining vet clearances, and facilitating animal pickup.

            Lastly, direct access to specific animal profiles or NGOs will not be supported, in line with the project’s objective to create an inclusive platform for all organizations, regardless of their popularity.
