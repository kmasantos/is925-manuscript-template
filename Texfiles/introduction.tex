\section{INTRODUCTION}
    \subsection{Background of the project}

            One of the most common names given to dogs in the Philippines is \textit{Bantay}, which means to guard\citep{Bantay}. This is rooted in how they were originally envisioned to be—the defender of the house, the first line of defense against burglars, and the one to take the bullet.

            In the first quarter of 2023, Social Weather Stations (SWS) , a prominent social research institution in the Philippines, conducted a survey to assess the prevalence of pet ownership among Filipino families. The survey revealed that a significant 64 percent of households own pets, with 59 percent of these households keeping dogs and/or cats. The remaining 5 percent of pet owners have other species of animals\citep{SwsHouseholdPet2023}. In addition to these findings, research also shows that as of May 2023, there were an estimated 13.11 million stray dogs and cats in the country or around 41 percent of their total population\citep{Overpopulation2024}. These animals are either abandoned family pets or the offspring of those that reproduced on the streets.
            
            Fortunately, the way people view animals, especially those that coexist with us in our homes, is slowly changing. In 1998, Republic Act No. 8485, also known as the Animal Welfare Act, was enacted into law\citep{RA8485}. This law aims to prevent cruelty to animals and ensure that they are treated humanely. It regulates various aspects of animal care, including proper housing, feeding, transportation, and the prohibition of animal cruelty and abuse. In 2013, Republic Act No. 10631 further strengthened these protections by imposing stronger penalties for violations of RA 8485 and explicitly including all forms of animal fighting—such as those conducted for entertainment—as acts of cruelty. This law also mandates the reporting of animal cruelty\citep{RA10631}.

            Over the years, various NGOs and self-supporting animal sanctuaries have been established to provide assistance to animals in need. These organizations not only raise awareness about animal rights but also serve as advocates for animals subjected to abuse\citep{KilluaNews}. In addition to their advocacy efforts, these organizations collaborate with policymakers to ensure that animals are protected from cruelty and exploitation. Their work plays a critical role in raising public awareness and influencing the development of stronger animal welfare laws. Despite this significant impact, these organizations largely rely on donations for funding, which serves as a limiting factor in their ability to expand operations, implement large-scale initiatives, and maintain long-term sustainability.
            
    \subsection{Objectives}
        This study aims to support NGOs and animal sanctuaries by designing and implementing an inclusive multi-platform solution that will help organizations effectively showcase animals in need and engage potential adopters or sponsors, ultimately enhancing visibility and increasing the chances of successful adoptions and support regardless of the organization's size and popularity.
        
        To achieve this, the research aims to develop an intuitive web application that enables organizations to easily create, update, and manage detailed rescue profiles. These profiles will include key information about the animals, such as their characteristics, needs, and stories. The mobile application, in turn, will offer users a personalized, matchmaking-style experience, providing tailored animal profile suggestions based on initial user inputs and refined through ongoing interactions with the platform. This approach will ensure that users receive increasingly relevant recommendations that align with their preferences.

        Additionally, the platform will undergo a comprehensive testing phase to ensure its functionality, usability, and overall performance. Unit tests will be implemented to verify the integrity of the code, with particular focus on critical logic and key features of the platform. In parallel, iterative user testing will be conducted to gather feedback from real users, identifying areas for improvement and refining the user experience. This continuous feedback loop will ensure the platform meets user expectations and performs reliably under various conditions.

        
    \subsection{Scope and Limitations}
        This study primarily focuses on the design and implementation of a multi-platform product aimed at helping Philippine-based animal welfare groups better connect with the rescue community for sponsorship and adoption.
        \subsubsection{Scope}
             The scope of this research project focuses on the development of a comprehensive platform designed to connect rescued animals with potential adopters or sponsors, with a particular emphasis on enhancing the matching process using machine learning and user interactions. The system will feature two primary components: a web application for NGOs to sign up and create detailed animal profiles, and a mobile application for users to discover and interact with these profiles. NGOs will be able to publish rescue profiles, including essential details such as animal characteristics, their stories, and adoption or sponsorship needs. Once animal profiles are published on the web platform, they will automatically be available on the mobile app, where users can browse and receive personalized 
             recommendations based on their preferences and prior interactions.

            The matching process will prioritize recommendations tailored to users who engage most strongly with specific animal characteristics or stories, refining the algorithm over time to improve the quality of matches. When a user swipes right on an animal profile, it will be saved to their matches, allowing them to view additional details such as the NGO that manages the animal, along with options to apply for adoption or sponsorship. For financial transactions, the mobile platform will support one-time credit card payments via Paymongo, facilitating sponsorship and other contributions.
        \subsubsection{Limitations}
            All other operations requiring manual intervention are out of scope, including but not limited to: establishing a bank account and Paymongo profile for donation disbursement, conducting interviews with adoption applicants, arranging in-person meetups, obtaining vet clearances, and facilitating animal pickup.

            Lastly, direct access to specific animal profiles or NGOs will not be supported, in line with the project’s objective to create an inclusive platform for all organizations, regardless of their popularity.
