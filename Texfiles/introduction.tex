\section{INTRODUCTION}
    \subsection{Background of the project}
            Your introduction must address the following questions (note that this is a guide, the questions must not be the subsections in the Introduction):
            \begin{itemize}
                \item What is the problem that you are trying to address? 
                \item Why is the problem interesting?
                \item What is the solution of others? Include a summary of what the others are currently doing.
                \item What is the background of the proposed solution?
                \item What is being proposed in this document? Include the following: summary of methodology and the expected results
            \end{itemize}
        \lipsum[4]
    \subsection{Objectives}
        What do you wish to achieve in your capstone project? Write one general and at least three specific objectives of your project. The objectives must:
            \begin{itemize}
                \item be clear, concise, declarative statement;
                \item provide direction to investigate the variables in the study;  
                \item focus on ways to measure the variables, to identify or describe them;
                \item identify relationships between variables;
                \item indicate results sought by the project proponent at the end of the process;
                \item be closely related to the statement of the problem; and
                \item be SMART: Specific, Measurable, Attainable, Realistic, Time-bound.
            \end{itemize}
    \subsection{Scope and Limitations}
        In a research or capstone project, the scope and limitations are essential elements that guide the work to stay focused and define the extent of the content to be covered. 
        \subsubsection{Scope}
            The scope of a study refers to the extent or boundary of the research and defines where the project will be deep-diving. It outlines what the project will encompass and also determines its size and the parameters to work within, including specific objectives, activities, resources, methodologies, and deliverables. For instance, if the project is about developing an AI model to predict stock prices, then the scope could include analyzing given datasets, pattern recognition, implementing machine learning algorithms, testing the model's accuracy, and more.\\
            In terms of scope, a clear definition is a vital step to avoid research deviations. It ensures that the project remains on track without straying into unrelated areas. If the scope is not well defined, the study can easily grow beyond your control.
        \subsubsection{Limitations}
            The limitations of a study are potential weaknesses or problems encountered during the study due to constraints on methodology or resources. These might include a lack of available or reliable data, time or financial constraints, lack of access to technology or software, and the assumptions you've had to make due to such constraints. \\
            For example, in the above-mentioned AI model, a limitation could be the unavailability of real-time stock data for accurate testing, or perhaps certain algorithms couldn't be used due to software limitations.